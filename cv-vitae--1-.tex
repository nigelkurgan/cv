%!TEX TS-program = xelatex
%!TEX encoding = UTF-8 Unicode
% Awesome CV LaTeX Template for CV/Resume
%
% This template has been downloaded from:
% https://github.com/posquit0/Awesome-CV
%
% Author:
% Claud D. Park <posquit0.bj@gmail.com>
% http://www.posquit0.com
%
%
% Adapted to be an Rmarkdown template by Mitchell O'Hara-Wild
% 23 November 2018
%
% Template license:
% CC BY-SA 4.0 (https://creativecommons.org/licenses/by-sa/4.0/)
%
%-------------------------------------------------------------------------------
% CONFIGURATIONS
%-------------------------------------------------------------------------------
% A4 paper size by default, use 'letterpaper' for US letter
\documentclass[11pt,a4paper,]{awesome-cv}

% Configure page margins with geometry
\usepackage{geometry}
\geometry{left=1.4cm, top=.8cm, right=1.4cm, bottom=1.8cm, footskip=.5cm}


% Specify the location of the included fonts
\fontdir[fonts/]

% Color for highlights
% Awesome Colors: awesome-emerald, awesome-skyblue, awesome-red, awesome-pink, awesome-orange
%                 awesome-nephritis, awesome-concrete, awesome-darknight

\definecolor{awesome}{HTML}{414141}

% Colors for text
% Uncomment if you would like to specify your own color
% \definecolor{darktext}{HTML}{414141}
% \definecolor{text}{HTML}{333333}
% \definecolor{graytext}{HTML}{5D5D5D}
% \definecolor{lighttext}{HTML}{999999}

% Set false if you don't want to highlight section with awesome color
\setbool{acvSectionColorHighlight}{true}

% If you would like to change the social information separator from a pipe (|) to something else
\renewcommand{\acvHeaderSocialSep}{\quad\textbar\quad}

\def\endfirstpage{\newpage}

%-------------------------------------------------------------------------------
%	PERSONAL INFORMATION
%	Comment any of the lines below if they are not required
%-------------------------------------------------------------------------------
% Available options: circle|rectangle,edge/noedge,left/right

\name{Nigel Kurgan}{}

\position{Postdoctoral Research Fellow}
\address{Novo Nordisk Foundation Center for Basic Metabolic Research}

\email{\href{mailto:nigel.kurgan@sund.ku.dk}{\nolinkurl{nigel.kurgan@sund.ku.dk}}}
\homepage{cbmr.ku.dk/research/research-groups/deshmukh-group/}
\orcid{0000-0002-5011-0297}
\googlescholar{Daqi2XQAAAAJ}
\github{nigelkurgan}
\twitter{nigelkurgan}

% \gitlab{gitlab-id}
% \stackoverflow{SO-id}{SO-name}
% \skype{skype-id}
% \reddit{reddit-id}

\quote{Exercise physiologist with expertise in metabolic assays,
proteomics, and bioinformatics. I'm interested in combining these
approaches to better understand interorgan communication mediated by
secreted proteins during exercise and in individuals with metabolic
diseases. I am an avid \texttt{R} user for data processing,
visualization, communication, reports/dashboards, and am passionate
about open and reproducible science.}

\usepackage{booktabs}

\providecommand{\tightlist}{%
	\setlength{\itemsep}{0pt}\setlength{\parskip}{0pt}}

%------------------------------------------------------------------------------



% Pandoc CSL macros
\newlength{\cslhangindent}
\setlength{\cslhangindent}{1.5em}
\newlength{\csllabelwidth}
\setlength{\csllabelwidth}{2em}
\newenvironment{CSLReferences}[2] % #1 hanging-ident, #2 entry spacing
 {% don't indent paragraphs
  \setlength{\parindent}{0pt}
  % turn on hanging indent if param 1 is 1
  \ifodd #1 \everypar{\setlength{\hangindent}{\cslhangindent}}\ignorespaces\fi
  % set entry spacing
  \ifnum #2 > 0
  \setlength{\parskip}{#2\baselineskip}
  \fi
 }%
 {}
\usepackage{calc}
\newcommand{\CSLBlock}[1]{#1\hfill\break}
\newcommand{\CSLLeftMargin}[1]{\parbox[t]{\csllabelwidth}{\honortitlestyle{#1}}}
\newcommand{\CSLRightInline}[1]{\parbox[t]{\linewidth - \csllabelwidth}{\honordatestyle{#1}}}
\newcommand{\CSLIndent}[1]{\hspace{\cslhangindent}#1}

\begin{document}

% Print the header with above personal informations
% Give optional argument to change alignment(C: center, L: left, R: right)
\makecvheader

% Print the footer with 3 arguments(<left>, <center>, <right>)
% Leave any of these blank if they are not needed
% 2019-02-14 Chris Umphlett - add flexibility to the document name in footer, rather than have it be static Curriculum Vitae
\makecvfooter
  {June 2023}
    {Nigel Kurgan~~~·~~~Curriculum Vitae}
  {\thepage~ of \pageref{LastPage}~}


%-------------------------------------------------------------------------------
%	CV/RESUME CONTENT
%	Each section is imported separately, open each file in turn to modify content
%------------------------------------------------------------------------------



\hypertarget{current-appointments}{%
\section{Current Appointments}\label{current-appointments}}

\begin{cventries}
    \cventry{Postdoctoral Research Fellow}{Novo Nordisk Foundation Center for Basic Metabolic Research, University of Copenhagen}{København, Denmark}{Sep 2022--Present}{}\vspace{-4.0mm}
\end{cventries}

\hypertarget{education}{%
\section{Education}\label{education}}

\begin{cventries}
    \cventry{Ph.D. Health Biosciences}{Brock University}{St. Catharines, Canada}{2022}{\begin{cvitems}
\item Conferred with Distinction and a Graduate Student Research Excellence Award
\item Funded by Scholarships from NSERC (2017-21) and QEII-GSST (2021-2022)
\item \href{https://dr.library.brocku.ca/handle/10464/16382}{Thesis: Sclerostin influences body composition adaptations to exercise training \faExternalLink}
\end{cvitems}}
    \cventry{M.Sc. Applied Health Sciences}{Brock University}{St. Catharines, Canada}{2017}{\begin{cvitems}
\item Funded by an Ontario Graduate Scholarship
\item \href{https://dr.library.brocku.ca/handle/10464/12927}{Thesis: Physical training, inflammation, and bone integrity in elite female rowers \faExternalLink}
\end{cvitems}}
    \cventry{B.Sc. Biomedical Sciences}{Brock University}{St. Catharines, Canada}{2015}{\begin{cvitems}
\item Conferred June 2015 with First-Class Standing
\item Thesis: Mitochondrial function and phospholipid composition changes in mdx mice skeletal muscle
\end{cvitems}}
\end{cventries}

\hypertarget{research-experience}{%
\section{Research Experience}\label{research-experience}}

\begin{cventries}
    \cventry{Research Assistant, Coorssen Top-down Proteomics Lab}{Brock University, Biology}{St. Catharines, Canada}{2017-20}{\begin{cvitems}
\item Top-down proteomic analysis, method development, and teaching
\end{cvitems}}
    \cventry{Research Assistant, Tsiani Cell Signaling Lab}{Brock University, Biology}{St. Catharines, Canada}{2018-20}{\begin{cvitems}
\item Optimized in vitro assays involving treatment of cancer cells with human serum
\end{cvitems}}
    \cventry{Laboratory Technician, Athlete Monitoring}{Canadian Sports Institute of Ontario}{London, Canada}{2016}{\begin{cvitems}
\item Blood and saliva collection and analysis from professional female athletes
\end{cvitems}}
    \cventry{Research Assistant, Brock-Niagara Center for Health and Well-Being}{Brock University, Kinesiology}{St. Catharines, Canada}{2011-2015}{\begin{cvitems}
\item Peer support for physical activity in special populations
\end{cvitems}}
    \cventry{Research Assistant, LebLanc Lab}{Brock University, Biology}{St. Catharines, Canada}{2013-14}{\begin{cvitems}
\item Optimication of mitochondrial isolations and respiration from skeletal muscle of mdx mice
\end{cvitems}}
\end{cventries}

\hypertarget{teaching}{%
\section{Teaching}\label{teaching}}

\footnotesize

In addition to.

\normalsize
\begin{cventries}
    \cventry{Brock University, Kinesiology}{Instructor}{St. Catharines, Canada}{2019-2020}{\begin{cvitems}
\item KINE 2P09 – Human Physiology (2019-20)
\end{cvitems}}
    \cventry{Brock University, Health Sciences and Kinesiology}{Teaching Assisstant}{St. Catharines, Canada}{2016-21}{\begin{cvitems}
\item KINE 2P09 – Human Physiology (2020-21)
\item KINE 2P97 – Exercise Metabolism (2021)
\item GERO 5P88 – The Process of Aging (2020)
\item KINE 1P98 – Musculoskeletal Anatomy (2018-19)
\item KINE 1P90 – Human Anatomy and Physiology (2017-18)
\item HLSC 1F90 – Introduction to Health Sciences (2016-18)
\end{cvitems}}
    \cventry{Brock University, Biology and Health Sciences}{Laboratory Demonstrator}{St. Catharines, Canada}{2015-20}{\begin{cvitems}
\item AHSC 7P96 – Top-Down Proteomics (2020)
\item HLSC 4P95 – Human Pathology (2016)
\item HLSC 3P02 – Introduction to Human Immunology (2015)
\end{cvitems}}
\end{cventries}

\hypertarget{supervision}{%
\section{Supervision}\label{supervision}}

\footnotesize

In addition to.

\normalsize
\begin{cventries}
    \cventry{University of Copenhagen, NNF CBMR}{M.Sc Student co-supervisor}{København, Denmark}{2023-24}{\begin{cvitems}
\item Designed and co-supervised a M.Sc. project on exercise regulated peptides that influence metabolism
\end{cvitems}}
    \cventry{Brock University, Kinesiology}{M.Sc Student co-supervisor}{St. Catharines, Canada}{2020-22}{\begin{cvitems}
\item Co-supervised a M.Sc. project on a bone derived protein that influences muscle fibre type
\end{cvitems}}
    \cventry{Brock University, Kinesiology}{Undergraduate Student co-supervisor}{St. Catharines, Canada}{2017-20}{\begin{cvitems}
\item Co-supervised an undergraduate research project on cytokine response to acute exercise
\end{cvitems}}
\end{cventries}

\hypertarget{leadership-service-community-engagement}{%
\section{Leadership, Service \& Community
Engagement}\label{leadership-service-community-engagement}}

\footnotesize

\emph{Manuscript peer-review}: \normalsize

\begin{cventries}
    \cventry{\textbf{\href{Conference Planning https://brocku.ca/bone-and-muscle-health/events/mhef-2021/ }{Conference Planning \faExternalLink}}: Brock University's Muscle Health and Education Forum (St. Catharines, Canada)}{}{}{2021}{\begin{cvitems}
\item Designed the program, recruited speakers, and chaired sessions
\end{cvitems}}
    \cventry{\textbf{Conference Symposia Chair}: Brock University's Muscle Health and Education Forum (St. Catharines, Canada)}{}{}{2018-21}{\begin{cvitems}
\item Free Communication chair – Health Science and Biotechnology, Mapping the New Knowledges Meetingx4 years
\end{cvitems}}
    \cventry{\textbf{Conference Symposia Chair}: OEP (Barrie, Canada)}{}{}{2016}{\begin{cvitems}
\item Sessional Chair – Powerhouse Physiology, Ontario Exercise Physiology Conference
\end{cvitems}}
    \cventry{\textbf{Conference Symposia Chair}: Brock University's Muscle Health and Education Forum (St. Catharines, Canada)}{}{}{2015}{\begin{cvitems}
\item Chair and Master of Ceremony – Brock’s Math and Science Undergraduate Research Conference
\end{cvitems}}
\end{cventries}

\hypertarget{additional-training-and-professional-development}{%
\section{Additional training and professional
development}\label{additional-training-and-professional-development}}

\begin{cvhonors}
    \cvhonor{}{\textbf{Course in Laboratory Animal Science EU Function ABD (felasa)} (University of Copenhagen, Department of Experimental Medicine)}{}{2022}
    \cvhonor{}{\textbf{DDA Reproducible Research in R – an advanced workshop on creating collaborative and automated analysis for PhD Students and Postdocs} (Danish Diabetes and Endocrine Academy (DDEA))}{}{2022}
    \cvhonor{}{\textbf{Just Bash It} (University of Copenhagen, Center for Health Data Science)}{}{2022}
    \cvhonor{}{\textbf{From Excel to R} (University of Copenhagen, Center for Health Data Science)}{}{2022}
    \cvhonor{}{\textbf{\href{https://www.linkedin.com/posts/ddeacademy_postdocs-postdoc-research-ugcPost-7008339764479152128-qIHj?utm_source=share&utm_medium=member_desktop}{DDA Postdoc Summit - Challenge \faExternalLink}} (Danish Diabetes and Endocrine Academy (DDEA))}{}{2022}
    \cvhonor{}{\textbf{Canadian Council of Animal Care and Use of Experimental Animals Certification} (Brock University, ACC)}{}{2020}
\end{cvhonors}

\hypertarget{skills}{%
\section{Skills}\label{skills}}

\begin{table}[!h]
\centering\begingroup\fontsize{8}{10}\selectfont

\begin{tabular}{lll}
\toprule
\textbf{Laboratory} & \textbf{Physiological/Technical} & \textbf{Software/Tools}\\
\midrule
Mitochondrial respiration & Metabolic testing (humans and rodents) & R\\
Bead-based multiplex fluorescent assays & Body composition (humans and rodents) & Rstudio\\
Extracellular vesicle isolations & Human phlebotomy & RMarkdown\\
Top-down proteomics & Muscle ultrasound & Git/Github\\
Bottom-up proteomics & Cell culture & SPSS\\
\addlinespace
Co-immunoprecipitatoin & IPGTT and ITT & GraphPad Prism\\
Immunoblotting & Tail vein injections & Luminex and Bioplex manager\\
ELISA & Mouse dissection & ImageJ/Image Lab\\
Kinetic protein assays & King Fisher robot & Delta2D\\
Histology (adipose and muscle) & NA & Spectronaut\\
\addlinespace
In vitro assays & NA & DIA-NN\\
NA & NA & MaxQuant\\
NA & NA & Perseus\\
NA & NA & MS Fragger\\
NA & NA & Cytoscape\\
\addlinespace
NA & NA & String\\
NA & NA & DAVID\\
\bottomrule
\end{tabular}
\endgroup{}
\end{table}

\newpage

\hypertarget{awards-and-funding}{%
\section{Awards and Funding}\label{awards-and-funding}}

\begin{cvhonors}
    \cvhonor{}{\textbf{NSERC Postdoctoral Fellowship} (University of Copenhagen): Awarded 90,000 CAD to study mechanisms of tissue crosstalk mediated by extracellular vesicles}{}{2023-25\newline~\newline}
    \cvhonor{}{\textbf{Distinguished Graduate Student} (Brock University): Awarded 100 CAD for being the most distinguished graduate from my graduate program}{}{2022\newline~\newline}
    \cvhonor{}{\textbf{CSEP Graduate Student Oral Presentation Finalist} (CSEP Conference): Awarded 250 CAD for being selected as a finalist for best graduate student oral presentation}{}{2021\newline~\newline}
    \cvhonor{}{\textbf{QEII Graduate Scholarship in Science and Technology} (Brock University): Awarded a 15,000 CAD scholarship for my Ph.D. work}{}{2020-21\newline~\newline}
    \cvhonor{}{\textbf{Jack M Miller Excellence in Research} (Brock University): Awarded 1,341 CAD in recognition for my excellence in researc}{}{2020\newline~\newline}
    \cvhonor{}{\textbf{NSERC Postgraduate Scholarship – Doctoral} (Brock University): Awarded 63,000 CAD to study bone-secreted proteins during exercise}{}{2017-19\newline~\newline}
    \cvhonor{}{\textbf{Dean of Graduate Studies Excellence Scholarship} (Brock University): Awarded 21,000 CAD for my academic excellence}{}{2017-19\newline~\newline}
    \cvhonor{}{\textbf{Dean of Graduate Studies Entrance Scholarship – Doctoral} (Brock University): Awarded 2,000 CAD for my academic excellence}{}{2017\newline~\newline}
    \cvhonor{}{\textbf{Ontario Graduate Scholarship – Masters} (Brock University): Awarded 15,000 CAD to study bone health in elite female rowers}{}{2016-17\newline~\newline}
    \cvhonor{}{\textbf{Dean of Graduate Studies Excellence Scholarship – Masters} (Brock University): Awarded 2,500 CAD for my academic excellence}{}{2016-17\newline~\newline}
    \cvhonor{}{\textbf{CSEP Poster Award Finalist} (CSEP Conference): Selected as a finalist for best poster}{}{2016\newline~\newline}
    \cvhonor{}{\textbf{Dean of Graduate Studies Spring Research Fellowship} (Brock University): Awarded 4,000 CAD for my academic excellence}{}{2016\newline~\newline}
    \cvhonor{}{\textbf{Dean of Graduate Studies Entrance Scholarship} (Brock University): Awarded 2,500 CAD for my academic excellence}{}{2015-16\newline~\newline}
    \cvhonor{}{\textbf{Graduate Studies Fellowship} (Brock University): Awarded 7,000 CAD to study the bone response to acute exercise}{}{2015\newline~\newline}
    \cvhonor{}{\textbf{Brock Returning Scholars Award/Deans Honour List} (Brock University): Awarded 1,500 CAD for my academic excellence}{}{2012-14\newline~\newline}
    \cvhonor{}{\textbf{Brock Entrance Scholars Award} (Brock University): Awarded 2,500 CAD for my academic excellence}{}{2010\newline~\newline}
\end{cvhonors}

\newpage

\hypertarget{publications}{%
\section{Publications}\label{publications}}

\footnotesize

*indicates equal contribution

\setlength{\leftskip}{0cm}

\textbf{2022}

\setlength{\leftskip}{1cm}

Charnaud, S., Munro, J., Semenec, L., Mazhari, R., Brewster, J., Bourke,
C., \textbf{Ruybal‐Pesántez, S.}, James, R., Lautu‐Gumal, D., Karuna‐
jeewa, H., \& Mueller, I. (2022, \emph{accepted}). \emph{PacBio
long‐read amplicon sequencing for scalable high‐resolution population
allele typing of the complex CYP2D6 locus}. Communications Biology.

\textbf{Ruybal‐Pesántez, S.}, Tiedje, K. E., Pilosof, S., Tonkin‐Hill,
G., He, Q., Rask, T. S., Amenga‐Etego, L., Oduro, A. R., Koram, K. A.,
Pas‐ cual, M., \& Day, K. P. (2022). \emph{Age‐specific patterns of DBLα
var diversity can explain why residents of high malaria transmission
areas remain susceptible to Plasmodium falciparum blood stage infection
throughout life.} International Journal for Parasitology.
\url{https://doi.org/10.1016/j.ijpara.2021.12.001}

\setlength{\leftskip}{2cm}

\textbf{- This work was featured on the Herminthology
\#WomenBehindTheWork initiative
\href{https://facebook.com/102760015458811/posts/151514423916703/?d=n}{\faExternalLink}}

Feng, Q., Tiedje, K. E., \textbf{Ruybal‐Pesántez, S.}, Tonkin‐Hill, G.,
Duffy, M. F., Day, K. P., Shim, H., \& Chan, Y. (2022). \emph{An
accurate method for identifying recent recombinants from unaligned
sequences.} Bioinformatics.
\url{https://doi.org/10.1093/bioinformatics/btac012}

\setlength{\leftskip}{0cm}

\textbf{2021}

\setlength{\leftskip}{1cm}

\textbf{Ruybal‐Pesántez, S.}, Sáenz, F., Deed, S. L., Johnson, E. K.,
Larremore, D. B., Vera-Arias, C. A., Tiedje, K. E. \& Day, K. P. (2021,
\emph{pre-print}). \emph{Clinical malaria incidence following an
outbreak in Ecuador was predominantly associated with Plasmodium
falciparum with recombinant variant antigen gene repertoires.} medRxiv.
\url{https://doi.org/10.1101/2021.04.12.21255093}

Mazhari, R., \textbf{Ruybal‐Pesántez, S.}, Angrisano, F.,
Kiernan‐Walker, N., Hyslop, S., Longley, R. J., Bourke, C., Chen, C.,
Williamson, D. A., Robinson, L. J., Mueller, I., \& Eriksson, E. M.
(2021). \emph{SARS‐CoV‐2 Multi‐Antigen Serology Assay}. Methods and
Protocols, 4(4), 72. \url{https://doi.org/10.3390/mps4040072}

Argyropoulos, D. C.* , \textbf{Ruybal‐Pesántez, S.}* , Deed, S. L.,
Oduro, A. R., Dadzie, S. K., Appawu, M. A., Asoala, V., Pascual, M.,
Koram, K. A., Day, K. P., \& Tiedje, K. E. (2021). \emph{The impact of
indoor residual spraying on Plasmodium falciparum microsatellite
variation in an area of high seasonal malaria transmission in Ghana,
West Africa}. Molecular Ecology, 30(16), 3974--3992.
\url{https://doi.org/10.1111/mec.16029}

\setlength{\leftskip}{2cm}

\textbf{- This work was chosen by the editors to be featured in the
Molecular Ecology blog
\href{https://molecularecologyblog.com/2021/09/01/interview-with-the-authors-does-indoor-spraying-alter-the-genetic-diversity-of-malaria-causing-parasites-and-what-does-this-mean-for-long-term-control/}{\faExternalLink}}

\setlength{\leftskip}{1cm}

Tonkin‐Hill, G., \textbf{Ruybal‐Pesántez, S.}, Tiedje, K. E., Rougeron,
V., Duffy, M. F., Zakeri, S., Pumpaibool, T., Harnyuttanakorn, P.,
Branch, O. H., Ruiz‐Mesía, L., Rask, T. S., Prugnolle, F., Papenfuss, A.
T., Chan, Y., \& Day, K. P. (2021). \emph{Evolutionary analyses of the
major variant surface antigen‐encoding genes reveal population structure
of Plasmodium falciparum within and between continents}. PLOS Genetics,
17(2), e1009269. \url{https://doi.org/10.1371/journal.pgen.1009269}

\setlength{\leftskip}{2cm}

\textbf{- This work was chosen by the editors to be featured with an
accompanying Perspectives piece
\href{https://doi.org/10.1371/journal.pgen.1009344}{\faExternalLink}}

\setlength{\leftskip}{0cm}

\textbf{2020}

\setlength{\leftskip}{1cm}

Narh, C. A., Ghansah, A., Duffy, M. F., \textbf{Ruybal‐Pesántez, S.},
Onwona, C. O., Oduro, A. R., Koram, K. A., Day, K. P.* , \& Tiedje, K.
E.* (2020). \emph{Evolution of antimalarial drug resistance markers in
the reservoir of Plasmodium falciparum infections in the Upper East
Region of Ghana}. The Journal of Infectious Diseases.
\url{https://doi.org/10.1093/infdis/jiaa286}

\setlength{\leftskip}{0cm}

\textbf{2019}

\setlength{\leftskip}{1cm}

Pilosof, S., He, Q., Tiedje, K. E., \textbf{Ruybal‐Pesántez, S.}, Day,
K. P., \& Pascual, M. (2019). \emph{Competition for hosts modulates vast
antigenic diversity to generate persistent strain structure in
Plasmodium falciparum.} PLOS Biology, 17(6), e3000336.
\url{https://doi.org/10.1371/journal.pbio.3000336}

\setlength{\leftskip}{0cm}

\textbf{2018}

\setlength{\leftskip}{1cm}

He, Q., Pilosof, S., Tiedje, K. E., \textbf{Ruybal‐Pesántez, S.},
Artzy‐Randrup, Y., Baskerville, E. B., Day, K. P., \& Pascual, M.
(2018). \emph{Networks of genetic similarity reveal non‐neutral
processes shape strain structure in Plasmodium falciparum}. Nature
Communications, 9(1), 1817.
\url{https://doi.org/10.1038/s41467-018-04219-3}

Rorick, M. M., Artzy‐Randrup, Y., \textbf{Ruybal‐Pesántez, S.}, Tiedje,
K. E., Rask, T. S., Oduro, A., Ghansah, A., Koram, K., Day, K. P., \&
Pascual, M. (2018). \emph{Signatures of competition and strain structure
within the major blood‐stage antigen of Plasmodium falciparum in a local
community in Ghana}. Ecology and Evolution, 8(7), 3574--3588.
\url{https://doi.org/10.1002/ece3.3803}

\setlength{\leftskip}{0cm}

\textbf{2017}

\setlength{\leftskip}{1cm}

\textbf{Ruybal‐Pesántez, S.}, Tiedje, K. E., Rorick, M. M.,
Amenga‐Etego, L., Ghansah, A., Oduro, A. R., Koram, K. A., \& Day, K. P.
(2017). \emph{Lack of Geospatial Population Structure Yet Significant
Linkage Disequilibrium in the Reservoir of Plasmodium falciparum in
Bongo District, Ghana}. The American Journal of Tropical Medicine and
Hygiene, 97(4), 1180--1189. \url{https://doi.org/10.4269/ajtmh.17-0119}

\textbf{Ruybal‐Pesántez, S.}* , Tiedje, K. E.* , Tonkin‐Hill, G., Rask,
T. S., Kamya, M. R., Greenhouse, B., Dorsey, G., Duffy, M. F., \& Day,
K. P. (2017). \emph{Population genomics of virulence genes of Plasmodium
falciparum in clinical isolates from Uganda}. Scientific Reports, 7(1),
11810. \url{https://doi.org/10.1038/s41598-017-11814-9}

\setlength{\leftskip}{0cm}

\newpage

\hypertarget{digital-tools}{%
\section{Digital tools}\label{digital-tools}}

\footnotesize

For other non-traditional academic contributions, I have also developed
several \texttt{R\ Shiny} web applications to support COVID-19
surveillance efforts and \texttt{R\ flexdashboard} for real-time updates
and data visualization of both programmatic/operational aspects and
preliminary epidemiological trends as part of the coordination of
population-based field studies.
\href{https://github.com/shaziaruybal}{Check out my GitHub
\faExternalLink}

\begin{cventries}
    \cventry{Shazia Ruybal-Pesántez}{\href{https://shaziaruybal.shinyapps.io/covidClassifyR}{CovidClassifyR \faExternalLink}}{}{Sep 2021}{\begin{cvitems}
\item This Shiny web application was developed to support COVID-19 serosurveillance in Papua New Guinea enabling classification of unknown samples as recently exposed to SARS-CoV-2. This tool makes the downstream processing, quality control and interpretation of the raw data generated from a validated COVID-19 serological assay \href{https://doi.org/10.3390/mps4040072}{(Mazhari et al 2020)} accessible to all researchers without the need for a specialist background in statistical methods and advanced programming. Funding was provided by a COVID-19 Digital Grant, \href{https://www.science.org.au/news-and-events/news-and-media-releases/regional-research-set-get-digital-boost}{media release \faExternalLink}
\end{cvitems}}
    \cventry{Raúl Fernández, Shazia Ruybal-Pesántez, Esteban Ortíz}{\href{https://covid19analytics.shinyapps.io/VaccinationScore/}{COVID-19 VaccinationScore \faExternalLink}}{}{Feb 2021}{\begin{cvitems}
\item This Shiny web application was developed during the initial vaccine roll-out in Ecuador to help individuals better understand their "priority status" to receive their COVID-19 vaccine. An algorithm was applied to calculate a priority score based on an individuals answers to a set of questions on socioeconomic status, occupation, exposure, risk behavior, comorbidities, etc. \href{https://www.elcomercio.com/actualidad/herramienta-recibir-vacuna-covid19-ecuador.html}{Newspaper article (in Spanish) \faExternalLink}
\end{cvitems}}
    \cventry{Shazia Ruybal-Pesántez}{Serosurveillance of COVID-19 in Ecuadorian blood donors dashboard (not open-source)}{}{Jun 2020}{\begin{cvitems}
\item This R flexdashboard was developed to support COVID-19 serosurveillance in Ecuadorian blood donors in collaboration with the Ecuadorian Red Cross National Blood Bank as part of the emergency response to COVID-19 (active early in the pandemic, June 2020-Dec 2020). This dashboard presented anonymized and aggregated data generated from monthly screening of blood donation samples to visualize seroprevalence trends. Due to confidentiality and internal permissions at ERC this tool is not publicly available.
\end{cvitems}}
    \cventry{Shazia Ruybal-Pesántez}{ICEMR weekly dashboard (not open-source)}{}{Mar 2020}{\begin{cvitems}
\item This R flexdashboard was developed to support the ICEMR field teams in Madang, Papua New Guinea (active during the entire longitudinal cohort study March 2020 until Sep 2021). This dashboard was updated weekly and presented operational data (e.g. follow-up rates in each field site) that could be used by the team to plan field activies and identify any areas for improvement as well as preliminary epidemiological trends (e.g. RDT positivity, prevalence of fever). As this tool was meant for internal use within the ICEMR project, it is not publicly available.
\end{cvitems}}
    \cventry{Shazia Ruybal-Pesántez}{processqpcR}{}{In development}{\begin{cvitems}
\item This Shiny web application is in development to support laboratory researchers with little to no programming skills with a tool for downstream processing of raw data generated from several qPCR machines (e.g. Lightcyler480, Quantstudio, etc). Functions will include automatic matching of sample IDs using a user-supplied 96-well or 384-well plate map, quantification of unknown samples using the assay standard curve/positive controls (e.g. to detect malaria-positive samples) and some preliminary visualizations of the data.
\end{cvitems}}
\end{cventries}

\hypertarget{selected-presentations}{%
\section{Selected presentations}\label{selected-presentations}}

\footnotesize

I have participated in oral and poster presentations at 18 conferences
(13 international, 5 national; 6 travel awards).

\begin{cvhonors}
    \cvhonor{}{\textbf{Basic Metabolic Research and Critical Thinking Ph.D. Seminar  – Lead seminars on the application of mass-spectrometry-based proteomics and metabolomics}, University of Copenhagen, Center for Protein Research}{}{2023\newline~\newline}
    \cvhonor{}{\textbf{HLSC 4P98 – Leveraging proteomics to understand the molecular response to exercise}, Brock University, Faculty of Biological Sciences}{}{2018-21\newline~\newline}
    \cvhonor{}{\textbf{GERO 5P88 – Molecular mechanisms of age-related bone loss, Annually}, Brock University, Faculty of Applied Health Sciences}{}{2019-20\newline~\newline}
    \cvhonor{}{\textbf{KINE 1P90 – Gas Exchange and Transport, Annually}, Brock University, Faculty of Applied Health Sciences}{}{2017-20\newline~\newline}
    \cvhonor{}{\textbf{KINE 1P90 – The Blood and its Constituents, Annually}, Brock University, Faculty of Applied Health Sciences}{}{2017-20\newline~\newline}
    \cvhonor{}{\textbf{HSLC 2P09 – Ion Channels and Action Potentials, Annually}, Brock University, Faculty of Applied Health Sciences}{}{2017-18\newline~\newline}
\end{cvhonors}

\hypertarget{about-me}{%
\section{About me}\label{about-me}}

I am half Ecuadorian and half American, born in The Netherlands, I am
fluent in English and Spanish (beginner French and Dutch), and grew up
overseas in several countries in Africa and Latin America: Ecuador,
Tanzania, Guatemala, and Honduras. I am an Australian Permanent Resident
and have been living in Melbourne, Australia since 2014 when I moved to
pursue my PhD. Apart from my research, I am highly committed to
furthering health and development initiatives, particularly in my home
country of Ecuador. From early 2016 until the COVID-19 pandemic, I was
CEO and co-founder of the The Artisan Project, a social enterprise that
worked hand in hand with talented indigenous artisans in Ecuador. We
used fashion as a tool to create income-generating opportunities,
particularly for indigenous women, and impulse social impact and
innovation. During the COVID-19 pandemic I was actively involved as a
consultant epidemiologist providing analyses on case and testing trends,
importation dynamics, Reff calculations, among others, to the Ecuadorian
National COVID-19 Emergency Response committee - most of this work
remains unpublished due to competing political agendas and turnover of
public health officials.


\label{LastPage}~
\end{document}
